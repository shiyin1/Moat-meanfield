%%%%%%%%%%%%%%%%%%%%%%%%%%%%%%%%%%%%%%%%%%%%%%%%%%%%%%%%%%%%%%%%%%%%%%%%
\documentclass[a4paper]{article}
\usepackage{type1cm}
\usepackage{amsmath}
\usepackage{rawfonts}
\usepackage{feynmp}
\usepackage{color}
\usepackage{xcolor}
\setlength{\unitlength}{1mm}
%%%%%%%%%%%%%%%%%%%%%%%%%%%%%%%%%%%%%%%%%%%%%%%%%%%%%%%%%%%%%%%%%%%%%%%%
\newcommand{\magnification}{1}
\newlength{\logoheight}
\newlength{\logowidth}
\newcounter{logo}
\setcounter{logo}{1}
\newwrite\dvipscmds
\newcommand{\shiplogo}[2]{%
  \setbox0\hbox{#1}%
  \logowidth=\wd0%
  \logoheight=\ht0%
  \multiply\logoheight by \magnification%
  \multiply\logowidth by \magnification%
  \edef\dvipscmd{dvips -E -p=\the\value{logo} -n1 %
     -O -1in,-1in -T \the\logowidth,\the\logoheight\space %
     -o #2.eps \jobname}%
  \immediate\write\dvipscmds{\dvipscmd}%
  \shipout\box0%
  \stepcounter{logo}}
\newcommand{\logoframe}[5]{%
  \leavevmode
  \hbox{\vbox{\vskip#2\par\hbox{\hskip#1#5\hskip#3}\par\vskip#4}}}
\newenvironment{logos}[1]%
  {\ignorespaces\immediate\openout\dvipscmds=#1.sh}%
  {\immediate\closeout\dvipscmds}
%%%%%%%%%%%%%%%%%%%%%%%%%%%%%%%%%%%%%%%%%%%%%%%%%%%%%%%%%%%%%%%%%%%%%%%%
%~~~~~~~~~~~~~~~~~~~~ Custom Commands ~~~~~~~~~~~~~
%##### units#
\newcommand{\eV}[0]{\ensuremath{~\mathrm{eV} }}
\newcommand{\keV}[0]{\ensuremath{~\mathrm{keV} }}
\newcommand{\MeV}[0]{\ensuremath{~\mathrm{MeV} }}
\newcommand{\GeV}[0]{\ensuremath{~\mathrm{GeV} }}
\newcommand{\TeV}[0]{\ensuremath{~\mathrm{TeV} }}
\newcommand{\nb}[0]{\ensuremath{~\mathrm{nb} }}
\newcommand{\pb}[0]{\ensuremath{~\mathrm{pb} }}
\newcommand{\fb}[0]{\ensuremath{~\mathrm{fb} }}

%##### Order \fmfpen{thin}
\newcommand{\order}[1]{\ensuremath{ \mathcal{O}( #1 ) }}

\newcommand{\setfmfoptions}[0]{\fmfset{curly_len}{2mm}\fmfset{wiggly_len}{3mm}\fmfset{arrow_len}{3mm} \fmfset{dash_len}{2mm}\fmfpen{thin}}
\newcommand{\equref}[1]{\ref{#1}}
\newcommand{\secref}[1]{\ref{#1}}
\newcommand{\tabref}[1]{\ref{#1}}
\newcommand{\figref}[1]{\ref{#1}}
\newcommand{\im}{\mathrm{i}}
\renewcommand{\emph}[1]{#1}

\newcommand{\qw}{q_{W}}
\newcommand{\muW}{\mu_{W}}
\newcommand{\alphas}{\alpha_{s}}
\newcommand{\abs}[1]{\left|#1\right|}
\newcommand{\intd}{\mathrm{d}}
\newcommand{\klam}[1]{\left(#1\right)}
\newcommand{\sh}{\hat{s}}
\newcommand{\md}{\mathrm{d}}
\newcommand{\bol}[1]{\boldsymbol{#1}}
\newcommand{\mr}[1]{\mathrm{#1}}
\newcommand{\OS}{\mathrm{OS}}	

% shorthands for greek letters
\def\al{\alpha}
\def\be{\beta}
\def\ga{\gamma}
\def\de{\delta}
\def\veps{\varepsilon}
\def\la{\lambda}
\def\si{\sigma}
\def\Ga{\Gamma}
\def\De{\Delta}
\def\La{\Lambda}

%physical particles
\def\mathswitchr#1{\relax\ifmmode{\mathrm{#1}}\else$\mathrm{#1}$\fi}
\newcommand{\PB}{\mathswitchr B}
\newcommand{\PS}{\mathswitchr S}
\newcommand{\PV}{\mathswitchr V}
\newcommand{\PW}{\mathswitchr W}
\newcommand{\PZ}{\mathswitchr Z}
\newcommand{\PA}{\mathswitchr A}
\newcommand{\Pg}{\mathswitchr g}
\newcommand{\PH}{\mathswitchr H}
\newcommand{\Pe}{\mathswitchr e}
\newcommand{\Pne}{\mathswitch \nu_{\mathrm{e}}}
\newcommand{\Pane}{\mathswitch \bar\nu_{\mathrm{e}}}
\newcommand{\Pnmu}{\mathswitch \nu_\mu}
\newcommand{\Pd}{\mathswitchr d}
%\newcommand{\Pf}{\mathswitchr f}
\newcommand{\Pf}{f}
\newcommand{\Ph}{\mathswitchr h}
\newcommand{\Pl}{\mathswitchr l}
\newcommand{\Pu}{\mathswitchr u}
\newcommand{\Ps}{\mathswitchr s}
\newcommand{\Pb}{\mathswitchr b}
\newcommand{\Pc}{\mathswitchr c}
\newcommand{\Pt}{\mathswitchr t}
\newcommand{\Pq}{\mathswitchr q}
\newcommand{\Pep}{\mathswitchr {e^+}}
\newcommand{\Pem}{\mathswitchr {e^-}}
\newcommand{\Pmum}{\mathswitchr {\mu^-}}
\newcommand{\PWp}{\mathswitchr {W^+}}
\newcommand{\PWm}{\mathswitchr {W^-}}
\newcommand{\PWpm}{\mathswitchr {W^\pm}}
\newcommand{\PWO}{\mathswitchr {W^0}}
\newcommand{\PZO}{\mathswitchr {Z^0}}

% particle masses
\def\mathswitch#1{\relax\ifmmode#1\else$#1$\fi}
\newcommand{\MB}{\mathswitch {M_\PB}}
\newcommand{\Mf}{\mathswitch {m_\Pf}}
\newcommand{\Ml}{\mathswitch {m_\Pl}}
\newcommand{\Mq}{\mathswitch {m_\Pq}}
\newcommand{\MS}{\mathswitch {M_\PS}}
\newcommand{\MV}{\mathswitch {M_\PV}}
\newcommand{\MW}{\mathswitch {M_\PW}}
\newcommand{\hMW}{\mathswitch {\hat M_\PW}}
\newcommand{\MWpm}{\mathswitch {M_\PWpm}}
\newcommand{\MWO}{\mathswitch {M_\PWO}}
\newcommand{\MA}{\mathswitch {\lambda}}
\newcommand{\MZ}{\mathswitch {M_\PZ}}
\newcommand{\MH}{\mathswitch {M_\PH}}
\newcommand{\Me}{\mathswitch {m_\Pe}}
\newcommand{\Mmy}{\mathswitch {m_\mu}}
\newcommand{\Mta}{\mathswitch {m_\tau}}
\newcommand{\Md}{\mathswitch {m_\Pd}}
\newcommand{\Mu}{\mathswitch {m_\Pu}}
\newcommand{\Ms}{\mathswitch {m_\Ps}}
\newcommand{\Mc}{\mathswitch {m_\Pc}}
\newcommand{\Mb}{\mathswitch {m_\Pb}}
\newcommand{\Mt}{\mathswitch {m_\Pt}}
\newcommand{\GW}{\mathswitch {\Gamma_\PW}}

% shorthands for SM parameters
\newcommand{\scrs}{\scriptscriptstyle}
\newcommand{\sw}{\mathswitch {s_{\scrs\PW}}}
\newcommand{\cw}{\mathswitch {c_{\scrs\PW}}}
\newcommand{\thw}{\mathswitch {\theta_{\scrs\PW}}}
%\newcommand{\sw}{\mathswitch {s_\PW}}
%\newcommand{\cw}{\mathswitch {c_\PW}}
%\newcommand{\thw}{\mathswitch {\theta_{\PW}}}
%\newcommand{\swbar}{\mathswitch {\bar s_\PW}}
\newcommand{\swbar}{\mathswitch {\bar s_{\scrs\PW}}}
\newcommand{\swfbar}{\mathswitch {\bar s_{\PW,\Pf}}}
\newcommand{\swqbar}{\mathswitch {\bar s_{\PW,\Pq}}}
\newcommand{\Qf}{\mathswitch {Q_\Pf}}
\newcommand{\Ql}{\mathswitch {Q_\Pl}}
\newcommand{\Qq}{\mathswitch {Q_\Pq}}
\newcommand{\vf}{\mathswitch {v_\Pf}}
\newcommand{\af}{\mathswitch {a_\Pf}}
\newcommand{\gesi}{\mathswitch {g_\Pe}^{\sigma}}
\newcommand{\gem}{\mathswitch {g_\Pe}^-}
\newcommand{\gep}{\mathswitch {g_\Pe}^+}
\newcommand{\GF}{\mathswitch {G_\mu}}

\newcommand{\QCD}{{\mathrm{QCD}}}
\newcommand{\QED}{{\mathrm{QED}}}
\newcommand{\LEP}{{\mathrm{LEP}}}
\newcommand{\SLD}{{\mathrm{SLD}}}
\newcommand{\SM}{{\mathrm{SM}}}
\newcommand{\born}{{\mathrm{Born}}}
\newcommand{\nf}{{\mathrm{nf}}}
\newcommand{\marrow}[5]{%
    \fmfcmd{style_def marrow#1
    expr p = drawarrow subpath (1/4, 3/4) of p shifted 6 #2 withpen pencircle scaled 0.4;
    label.#3(btex #4 etex, point 0.5 of p shifted 6 #2);
    enddef;}
    \fmf{marrow#1,tension=0}{#5}}

\newcommand{\cut}[2]{%
    \fmfcmd{%
	vardef cross_bar (expr p, len, ang) =
	((-len/2,0)--(len/2,0))
	rotated (ang + angle direction length(p)/2 of p)
	shifted point length(p)/2 of p
	enddef;
	style_def crossed expr p =
	draw (wiggly p);
	ccutdraw cross_bar (p, 10mm, 90);
	enddef;}
    \fmf{crossed,fore=red}{#1,#2}}


\begin{document}
\begin{fmffile}{fmf}
\begin{logos}{pictures}
%

%\shiplogo{%
%	\begin{fmfgraph*}(35,35)
%		\fmfcmd{
%			% Please let me know if there’s a more efficient way to do this
%			path quadrant, q[], otimes;
%			quadrant = (0, 0) -- (0.5, 0) & quartercircle & (0, 0.5) -- (0, 0);
%			for i=1 upto 4: q[i] = quadrant rotated (45 + 90*i); endfor
%			otimes = q[1] & q[2] & q[3] & q[4] -- cycle;
%		}
%		\fmfwizard
%		\fmftop{l1,i,r1}
%		\fmfbottom{l2,o,r2}
%		\fmf{phantom,tension=5}{i,v1}
%		\fmf{phantom,tension=5}{v2,o}
%		\fmf{phantom,left,tension=0.4}{v1,v2,v1}
%		\fmfi{plain_arrow,fore=(0.392,,0.584,,0.929)}{fullcircle scaled .7w shifted (.5w,.5h)}
%		\fmfi{plain_arrow,fore=(0.392,,0.584,,0.929)}{fullcircle scaled .8w shifted (.5w,.5h)}
%		\fmfv{decor.shape=otimes,decor.filled=empty,decor.size=0.15w}{v1}
%	\end{fmfgraph*}
%}{diquark}

%\shiplogo{%
%	\begin{fmfgraph*}(35,35)
%		\fmfcmd{
%			% Please let me know if there’s a more efficient way to do this
%			path quadrant, q[], otimes;
%			quadrant = (0, 0) -- (0.5, 0) & quartercircle & (0, 0.5) -- (0, 0);
%			for i=1 upto 4: q[i] = quadrant rotated (45 + 90*i); endfor
%			otimes = q[1] & q[2] & q[3] & q[4] -- cycle;
%		}
%		\fmfwizard
%		\fmftop{l1,i,r1}
%		\fmfbottom{l2,o,r2}
%		\fmf{phantom,tension=5}{i,v1}
%		\fmf{phantom,tension=5}{v2,o}
%		\fmf{phantom,left,tension=0.4}{v1,v2,v1}
%		\fmfi{plain_arrow,fore=(0.392,,0.584,,0.929)}{reverse fullcircle scaled .7w shifted (.5w,.5h)}
%		\fmfi{plain_arrow,fore=(0.392,,0.584,,0.929)}{fullcircle scaled .8w shifted (.5w,.5h)}
%		\fmfv{decor.shape=otimes,decor.filled=empty,decor.size=0.15w}{v1}
%	\end{fmfgraph*}
%}{bilinear}

%\shiplogo{%
%	\begin{fmfgraph*}(35,35)
%		\fmfcmd{
%			% Please let me know if there’s a more efficient way to do this
%			path quadrant, q[], otimes;
%			quadrant = (0, 0) -- (0.5, 0) & quartercircle & (0, 0.5) -- (0, 0);
%			for i=1 upto 4: q[i] = quadrant rotated (45 + 90*i); endfor
%			otimes = q[1] & q[2] & q[3] & q[4] -- cycle;
%		}
%		\fmfwizard
%		\fmftop{l1,i,r1}
%		\fmfbottom{l2,o,r2}
%		\fmf{phantom,tension=5}{i,v1}
%		\fmf{phantom,tension=5}{v2,o}
%		\fmf{phantom,left,tension=0.4}{v1,v2,v1}
%		\fmfi{dashes,fore=(0,,0,,0.8)}{fullcircle rotated 0 scaled .775w shifted (.5w,.5h)}
%		\fmfv{decor.shape=otimes,decor.filled=empty,decor.size=0.15w}{v1}
%	\end{fmfgraph*}
%}{bilinear_alternative}

%\shiplogo{%
%	\begin{fmfgraph*}(35,35)
%		\fmfcmd{
%			% Please let me know if there’s a more efficient way to do this
%			path quadrant, q[], otimes;
%			quadrant = (0, 0) -- (0.5, 0) & quartercircle & (0, 0.5) -- (0, 0);
%			for i=1 upto 4: q[i] = quadrant rotated (45 + 90*i); endfor
%			otimes = q[1] & q[2] & q[3] & q[4] -- cycle;
%		}
%		\fmfwizard
%		\fmftop{l1,i,r1}
%		\fmfbottom{l2,o,r2}
%		\fmf{phantom,tension=5}{i,v1}
%		\fmf{phantom,tension=5}{v2,o}
%		\fmf{phantom,left,tension=0.4}{v1,v2,v1}
%		\fmfi{dashes,fore=(0,,0,,0)}{fullcircle rotated 0 scaled .775w shifted (.5w,.5h)}
%		\fmfv{decor.shape=otimes,decor.filled=empty,decor.size=0.15w}{v1}
%	\end{fmfgraph*}
%}{ghost}

%\shiplogo{%
%	\begin{fmfgraph*}(35,35)
%		\fmfcmd{
%			% Please let me know if there’s a more efficient way to do this
%			path quadrant, q[], otimes;
%			quadrant = (0, 0) -- (0.5, 0) & quartercircle & (0, 0.5) -- (0, 0);
%			for i=1 upto 4: q[i] = quadrant rotated (45 + 90*i); endfor
%			otimes = q[1] & q[2] & q[3] & q[4] -- cycle;
%		}
%		\fmfwizard
%		\fmftop{l1,i,r1}
%		\fmfbottom{l2,o,r2}
%		\fmf{phantom,tension=5}{i,v1}
%		\fmf{phantom,tension=5}{v2,o}
%		\fmf{phantom,left,tension=0.4}{v1,v2,v1}
%		\fmfi{dots,fore=(0,,0,,0)}{fullcircle rotated 1.5 scaled .775w shifted (.5w,.5h)}
%		\fmfv{decor.shape=otimes,decor.filled=empty,decor.size=0.15w}{v1}
%	\end{fmfgraph*}
%}{ghost_alternative}

%\shiplogo{
%	\begin{fmfgraph*}(35,35)
%		\fmfcmd{
%			% Please let me know if there’s a more efficient way to do this
%			path quadrant, q[], otimes;
%			quadrant = (0, 0) -- (0.5, 0) & quartercircle & (0, 0.5) -- (0, 0);
%			for i=1 upto 4: q[i] = quadrant rotated (45 + 90*i); endfor
%			otimes = q[1] & q[2] & q[3] & q[4] -- cycle;
%		}
%		\fmfwizard
%		\fmftop{l1,i,r1}
%		\fmfbottom{l2,o,r2}
%		\fmf{phantom,tension=5}{i,v1}
%		\fmf{phantom,tension=5}{v2,o}
%		\fmf{phantom,left,tension=0.4}{v1,v2,v1}
%		\fmfi{plain, fore=(0,,0,,0)}{fullcircle scaled .75w shifted (.5w,.5h)}
%		\fmfv{decor.shape=otimes,decor.filled=empty,decor.size=0.15w}{v1}
%	\end{fmfgraph*}
%}{quark}

%\shiplogo{%
%	\begin{fmfgraph*}(35,35)
%		\fmfcmd{
%			% Please let me know if there’s a more efficient way to do this
%			path quadrant, q[], otimes;
%			quadrant = (0, 0) -- (0.5, 0) & quartercircle & (0, 0.5) -- (0, 0);
%			for i=1 upto 4: q[i] = quadrant rotated (45 + 90*i); endfor
%			otimes = q[1] & q[2] & q[3] & q[4] -- cycle;
%		}
%		\fmfwizard
%		\fmftop{l1,i,r1}
%		\fmfbottom{l2,o,r2}
%		\fmf{phantom,tension=5}{i,v1}
%		\fmf{phantom,tension=5}{v2,o}
%		\fmf{phantom,left,tension=0.4}{v1,v2,v1}
%		\fmfi{gluon, fore=(0.933,,0.463,,0)}{fullcircle scaled .75w rotated 90 shifted (.5w,.5h)}
%		\fmfv{decor.shape=otimes,decor.filled=empty,decor.size=0.15w}{v1}
%	\end{fmfgraph*}
%}{gluon}


\shiplogo{%
%	\begin{fmfgraph*}(50,40)
	\begin{fmfgraph*}(50,30)
		\fmfcmd{
			% Please let me know if there’s a more efficient way to do this
			path quadrant, q[], otimes;
			quadrant = (0, 0) -- (0.5, 0) & quartercircle & (0, 0.5) -- (0, 0);
			for i=1 upto 4: q[i] = quadrant rotated (45 + 90*i); endfor
			otimes = q[1] & q[2] & q[3] & q[4] -- cycle;
		}
		\fmfwizard
                
%             \fmfset{dash_len}{2mm}
	        \fmfpen{thick}
        		\fmfleft{i}
        		\fmfright{o}
		
%		\fmfv{label=$a_1\mu_1$,l.a=-60,l.d=0.1w}{i}
%		\fmfv{label=$a_2\mu_2$,l.a=-120,l.d=0.1w}{o}
		
		\fmf{double,foreground=(0,,0.1,,1),tension=3,label.side=left}{i,v1}
		\fmf{double,foreground=(0,,0.1,,1),tension=3}{v2,o}
        		\fmf{quark,fore=black,left=1}{v1,v2}
       		\fmf{quark,left=1}{v2,v1}
		
		\fmfv{decor.shape=circle,decor.filled=1,decor.size=3thick,b=(0.5,,0.5,,0.5)}{v1,v2}
%		\fmfv{decor.shape=otimes,decor.filled=empty,decor.size=0.15w}{v1}
        \end{fmfgraph*}
}{pi2-quark}



\shiplogo{%
%	\begin{fmfgraph*}(50,40)
	\begin{fmfgraph*}(50,30)
		\fmfcmd{
			% Please let me know if there’s a more efficient way to do this
			path quadrant, q[], otimes;
			quadrant = (0, 0) -- (0.5, 0) & quartercircle & (0, 0.5) -- (0, 0);
			for i=1 upto 4: q[i] = quadrant rotated (45 + 90*i); endfor
			otimes = q[1] & q[2] & q[3] & q[4] -- cycle;
		}
		\fmfwizard
                
%             \fmfset{dash_len}{2mm}
	        \fmfpen{thick}
        		\fmfleft{i}
        		\fmfright{o}
		
%		\fmfv{label=$a_1\mu_1$,l.a=-60,l.d=0.1w}{i}
%		\fmfv{label=$a_2\mu_2$,l.a=-120,l.d=0.1w}{o}
		
		\fmf{dashes,tension=3,label.side=left}{i,v1}
		\fmf{dashes,tension=3}{v2,o}
        		\fmf{dashes,fore=black,left=1}{v1,v2}
       		\fmf{dashes,left=1}{v2,v1}
		
		\fmfv{decor.shape=circle,decor.filled=empty,decor.size=6thick,b=(0.5,,0.5,,0.5)}{v1,v2}
%		\fmfv{decor.shape=otimes,decor.filled=empty,decor.size=0.15w}{v1}
        \end{fmfgraph*}
}{pi2-meson}


\shiplogo{%
	\begin{fmfgraph*}(40,30)

	        \fmfpen{thick}
%        		\fmfleft{i}
%        		\fmfright{o}
                \fmfbottom{i,o}
                \fmftop{t}
		
%		\fmfv{label=$a$,,$\mu$,l.a=-60,l.d=0.1w}{i}
%		\fmfv{label=$b$,,$\nu$,l.a=-120,l.d=0.1w}{o}
		
		
 		\fmf{dashes,tension=1}{i,v}
 		\fmf{dashes,tension=1}{v,o}
% 		\fmf{gluon,tension=1/2}{v,v}		
%        		\fmf{gluon,right=1,tension=1/2}{v,v}

                \fmffreeze  
                
                 \fmf{phantom,tension=5}{t,v1}
        		\fmf{dashes,right=1}{v,v1}
        		\fmf{dashes,right=1}{v1,v}

 %        		\fmf{ghost,left=1,tension=1,label=$a_1\quad(q-p)\quad b_1$}{v1,v2}
 %       		\fmf{gluon,fore=blue,left=1,tension=1,label=\color{red}{$a_4\mu_4\quad\longrightarrow (q-p)\quad a^{\prime}_4\mu^{\prime}_4$}}{v1,v2}
 %       		\fmf{ghost,left=1,tension=1,label=$a_2\qquad q\qquad b_2$}{v2,v1}
		
		\fmfv{decor.shape=circle,decor.filled=empty,decor.size=6thick,b=(0.5,,0.5,,0.5)}{v}
%		\fmfv{decor.shape=otimes,decor.filled=empty,decor.size=0.15w}{v1}
        \end{fmfgraph*}
}{pi2-meson-tadpole}




\shiplogo{%
%	\begin{fmfgraph*}(50,40)
	\begin{fmfgraph*}(40,20)
		\fmfcmd{
			% Please let me know if there’s a more efficient way to do this
			path quadrant, q[], otimes;
			quadrant = (0, 0) -- (0.5, 0) & quartercircle & (0, 0.5) -- (0, 0);
			for i=1 upto 4: q[i] = quadrant rotated (45 + 90*i); endfor
			otimes = q[1] & q[2] & q[3] & q[4] -- cycle;
		}
		\fmfwizard
                
%             \fmfset{dash_len}{2mm}
	        \fmfpen{thick}
        		\fmfleft{i}
        		\fmfright{o}
		
%		\fmfv{label=$a_1\mu_1$,l.a=-60,l.d=0.1w}{i}
%		\fmfv{label=$a_2\mu_2$,l.a=-120,l.d=0.1w}{o}
		
%		\fmf{dashes,tension=3,label=$\longleftarrow p$,label.side=left}{i,v1}
%		\fmf{dashes,tension=3}{v2,o}
%        		\fmf{quark,fore=black,left=1,label=$(q-p)$}{v1,v2}
%       		\fmf{quark,left=1,label=$q$}{v2,v1}
                \fmf{double,foreground=(0,,0.1,,1),tension=1,width=0.8thick}{v1,i}
                \fmf{double,foreground=(0,,0.1,,1),tension=1,width=0.8thick}{o,v1}

	       \fmfv{decor.shape=circle,decor.filled=empty,decor.size=10thick,b=(0.5,,0.5,,0.5)}{v1}

%                \fmfv{label=$q$,l.a=160,l.d=0.15w}{v1}

%                \fmffreeze                
%                \fmfiv{label=$c$,l.a=20,l.d=0.15w}{vloc(__v1)}


%		\fmfdot{v1,v2}
%		\fmfv{decor.shape=otimes,decor.filled=empty,decor.size=0.15w}{v1}
        \end{fmfgraph*}
}{Gam2-meson}

\shiplogo{%
%	\begin{fmfgraph*}(50,40)
	\begin{fmfgraph*}(40,20)
		\fmfcmd{
			% Please let me know if there’s a more efficient way to do this
			path quadrant, q[], otimes;
			quadrant = (0, 0) -- (0.5, 0) & quartercircle & (0, 0.5) -- (0, 0);
			for i=1 upto 4: q[i] = quadrant rotated (45 + 90*i); endfor
			otimes = q[1] & q[2] & q[3] & q[4] -- cycle;
		}
		\fmfwizard
                
%             \fmfset{dash_len}{2mm}
	        \fmfpen{thick}
        		\fmfleft{i}
        		\fmfright{o}
		
%		\fmfv{label=$a_1\mu_1$,l.a=-60,l.d=0.1w}{i}
%		\fmfv{label=$a_2\mu_2$,l.a=-120,l.d=0.1w}{o}
		
%		\fmf{dashes,tension=3,label=$\longleftarrow p$,label.side=left}{i,v1}
%		\fmf{dashes,tension=3}{v2,o}
%        		\fmf{quark,fore=black,left=1,label=$(q-p)$}{v1,v2}
%       		\fmf{quark,left=1,label=$q$}{v2,v1}
                \fmf{double,foreground=(0,,0.1,,1),tension=1,width=0.8thick}{v1,i}
                \fmf{double,foreground=(0,,0.1,,1),tension=1,width=0.8thick}{o,v1}

	       %\fmfv{decor.shape=circle,decor.filled=empty,decor.size=10thick,b=(0.5,,0.5,,0.5)}{v1}

%                \fmfv{label=$q$,l.a=160,l.d=0.15w}{v1}

%                \fmffreeze                
%                \fmfiv{label=$c$,l.a=20,l.d=0.15w}{vloc(__v1)}


%		\fmfdot{v1,v2}
%		\fmfv{decor.shape=otimes,decor.filled=empty,decor.size=0.15w}{v1}
        \end{fmfgraph*}
}{Gam2-meson-bare}



\end{logos}
\end{fmffile}
\end{document}
